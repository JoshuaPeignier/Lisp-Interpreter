\documentclass[a4paper,11pt]{article}
\usepackage[french]{babel} 
\usepackage{microtype}

\usepackage[utf8]{inputenc}
\usepackage{array}
\usepackage{amsmath} 
\usepackage{amssymb}
\usepackage{amsfonts}
\usepackage{amsthm}
\usepackage{fullpage}%
\usepackage[T1]{fontenc}%

\usepackage{graphicx}%
\usepackage{url}%
\usepackage{abstract}%

\usepackage{mathpazo}%
\parskip=0.5\baselineskip


\title{Interpréteur LISP}
\author{J. Peignier \& E. Varloot}
\date{24 Avril 2016}


\begin{document}

\maketitle

\tableofcontents

\pagebreak


\begin{abstract}
	Dans cet article, nous présentons un interprète Lisp en C++, écrit à partir de l'amélioration d'un code de base qui gère déjà le coeur de l'interprète (notamment les fonctions \texttt{eval} et \texttt{apply}) ; l'objectif est d'y intégrer toutes les subroutines habituelles du langage Lisp, ainsi qu'un dispositif de trace pour visualiser l'évaluation.
  %Ici votre résumé. Pas plus de 10 lignes en double colonnes, 5 en
  %simple colonne. Tous les mots-clés doivent y être cités. 
  \begin{description}
  \item[Mots-clés:] Interprète Lisp en C++ ; subroutine ; gestion d'erreurs
%\item[Classification ACM:] Voir les codes à l'adresse
  %\begin{center}
   % \url{http://www.acm.org/about/class/class/2012}.
  %\end{center}

  \end{description}
\end{abstract}


\section{Introduction}
	
	Le thème de ce projet consiste en la réalisation d'un interprète Lisp en C++, à partir d'un code de base fourni en C++ (capable d'évaluer une expression donnée à l'interprète, ainsi que de gérer l'environnement), ainsi qu'à l'aide d'un code complet en CamL d'un interprète Lisp à liaison dynamique, dont nous nous sommes inspirés pour le choix de subroutines à implémenter.
	Dans un souci de simplicité, nous nous intéressons grandement à l'implémentation des subroutines et au rattrapage des erreurs d'évaluation.
	Nous avons donc dû chercher toutes les sources d'erreur possible pour pouvoir les rattraper, et nous assurer que les erreurs d'évaluation n'auraient pas d'impact sur l'environnement courant.
	Nous nous sommes donc d'abord concentrés sur l'implémentation des subroutines en nous inspirant du code de l'interprète écrit en CamL, puis avons cherché pour chacune d'entre elles les erreurs possibles, ainsi que celles qui pourraient arriver dans la fonction d'évaluation du code de base.

\section{Implémentation}

	\subsection{Toplevel}
	Le code fourni en C++ présentait déjà une implémentation de \textit{toplevel}. Celui-ci se constitue d'une boucle principale qui attend une entrée de la part d'un utilisateur. Celle-ci est transmise à un \textit{lexer}, qui vérifie que la syntaxe de l'expression présentée est correcte, puis à un \textit{parser}, qui construit l'objet Lisp correspondant à l'expression entrée. Puis l'évaluation est lancée dans l'environnement global, et le résultat de celle-ci est affichée. Nous avons amélioré le programme pour faire en sorte que le résultat ne s'affiche pas lorsqu'une erreur a lieu à l'évaluation, car le comportement de l'interprète à ce moment-là est plus ou moins indéterminé.
	 
	 La seconde partie conserne la directive "setq" qui permet d'ajouté un élèment a l'environement global. Pour se faire nous avons assimilé "setq" a un element qu'on evalue. Lorsqu'on le trouve on evalue le secound argument de setq puis on la renvoie. De plus on étant l'environement general tel que le premier argument represente maintenant le resultat de l'evaluation du second.
	Nous avons rajouté un message dans le cas ou setq serait appelé plusieur fois lors d'une évaluation, en effet bien que cela ne pose pas de problème pour notre code, des cas patologique peuvent apparaitre en parliculier lorsque setq s'appelle lui-même.

\section{Subroutine}
\subsection{Gestion general}
	Les subroutines ne sont pas des éléments a évalué on les appel donc dans apply. Ainsi lorsque le code d'eval determine qu'un element ne peut être evalué et il essaye de l'appliqué a la place. Il appelle apply qui détermine grâce a une fonction is\_subr si il s'agis d'une sobroutine. Si c'est le cas on va chercher dans subr.cc qu'on a créé la fonction liée a la subroutine en question.

	Dans notre code d'origine, seul + et * était déjà fourni. Il nous a donc fallut écrire d'autre subroutine. Pour chaque subroutine on suit le modèle suivant, on vérifie que le nombre l'argument est coherent avec l'application de la subroutine puis on effectue les instructions necessaire a l'application de la subroutine. Si la subroutine est sensé renvoyé quelquechose, on le renvois sinon on renvois nil, ceci car tel qu'elle est écrit apply doit renvoyer quelquechose.


Les subroutines de notre interpreteur lisp sont:
\begin{itemize}
	\item[+] : (+ n m) renvois la somme des entiers n et m (était fourni dans le code initial)
	\item[*] : ... (était fourni dans le code initial)
	\item[-] : ...
	\item[=] : ...
	\item[cdr] : (cdr l) renvois la queu de la liste l
	\item[car] : (car l) renvois la tête (premier élèment) de la liste l.
	\item[cons] : (cons x l) créé la liste de tête x et de queue l.
	\item[read] : renvois ce qu'on entre dans le terminal (on utilise souvant setq read afin de prendre en argument de la même manière qu'un scanf)
	\item[print] : sert a afficher "Hello World!"
	\item[new\_line] : sert a passer a la ligne
	\item[numberp] : (numberp p) renvois t si p est un entier et nil sinon.
	\item[null] : (null l) renvois t si l est l'élèment nil et nil sinon.	
	\item[symbolp] : (symbolp q) renvois t si q est un symbole et nil sinon.
	\item[stringp] : (stringp s) renvois t si s est une chaine de charactère et nil sinon.
	\item[listp] : (listp p) renvois t si p est une liste et nil sinon.
	\item[debug] : debug permet d'afficher la trace d'évaluation et d'application de la fonction, utilisé srtout pour trouvé les erreurs d'évaluations. Debug sera revu dans la partie trace du rapport.
	\item[concat] : concatène l'ensemble des arguments qu'il reçois ( il lui en faut au moins un) : concat ne peut pas prendre une liste (ou nil) comme argument.
	\item[eval] : renvois le resultat de eval (de eval.cc) sur l'argument de cette subroutine
	\item[apply] : renvois le resultat de apply (de eval.cc) du premier argument sur la liste non evalué (avant apply) donné en second argument. (Il ne faut donc pas oublié d'utilisé quote).
	\item[error] : arrete le programe et renvois une erreur sert quand on ecrit des programme en LISP et qu'on souhaite faire des asserts.
	\item[end] : permet de mettre fin a notre interfae LISP, serait très interessant si ce n'était pas déjà le cas a chaque erreur.
	
	
\end{itemize}

\subsection{Cas particulier}
	Pour la subroutine "read" on a reutilisé le parxer car la fonction read était déjà défini dans celui-ci.
	
	Pour la subroutine "apply" il nous a fallu reprendre un peu le parser, voir si-dessous.

	Pour la subroutine "concat" il faut changer les entiers et symbols en chaine de caractères. Par simplification, on a copié une source internet (donné en reference) pour , pour changé nos entier.
	
	La subroutine "end" affiche "May LISP be with you!", cette reference nous parait mieux placé au debut du programme qu'as la fin.

	La subroutine "error" appel une fonction Lisp\_error qui effectue l'ensemble des actions attribué à "error". Nous avons pris le parti d'agir ainsi, afin de remplacer tout les messages d'erreur de notre code par des message d'error du même type que ceux de la subroutine "error". On a aussi modifié le messages d'erreur pour que les fonctions qui renvois des erreur n'affiche rien d'autre.

\section{Trace}
	La subroutine debug permet de voir les traces des évaluation des instructions qui suivent. Debug renvois nil se qui permet d'éviter des erreur pathologiques.
	
	Debug prend en argument un entier ou nil. Dans l'absolu on voudrait que debug nil arrete l'affichage de la trace, que debug n renvois la trace où l'affichage de l'evaluation soit de profondeur n en parenthèse et que dans le cas où n ne serait pas scrictement positif, on afficherait l'integralité de la trace.
	
\section{Quote}
	Avec un parser correct '(5 2) se transformerait en (quote (5 2)), ainsi (car '(5 2)) devrait renvoyer 5.	Grâce au commentaire de Solène Miraze sur le Piazza de la classe, une erreur qui empechait d'utiliser l'abreviation à été corrigé. 
	
	Quote est l'une des directive les plus importante de LISP car sans elle il est impossible d'indiquer a apply quand on a affaire a un element de type liste qu'on ne peut pas appliquer. On a donc eu de nombreux message d'erreur.
	
	Pour contourné se problème, on a dans une version antèrieur de notre code decrété que si le premier element d'un parentésage est un nombre alors on a affaire a une liste et on la renvois tel quel. Cette étape n'avais qu'un but temporaire (après tout (3 + 5) n'est pas une liste juste une erreur). Un language issus de LISP pourrais utilisé cette idée (avec quelque verification suplementaire) pour simplifié l'écriture parfois lourde du LISP. 
	
	En effet on chercher a éviter de devoir réécrire quote a chaque fois.
	
\section{Nouveau modèle d'Environement}
	L'environement est défini comme un vecteur de couple (nom, valeur). A chaque fois qu'on rajoute un élément à l'environement (directive setq) on modifie la taille de l'environement afin de prendre en compte le nouvel arrivant. Bien qu'on place cet élément a la fin du vecteur du toute recherche commence par lui. En d'autre mot on remonte le vecteur jusqu'as trouvé se qu'on cherche. Par  On souhaiterait utilisé une liste chainéé plutôt qu'un vecteur. Car cette structure est plus adapté a ce modèle, on rajoute les nouveaux éléménts au debut de la liste (en temps constant) et on effectue nos recherche d'élément de manière recursif sur les listes.
	
	Pour effectuer ses changements on modifie env.cc (et env.hh); la classe Binding (les couples) n'as pas besoin de changement (et ses fonctions non plus) on va donc modifié la classe Environement et les fonctions de cette classe environement.
	
\section{Conclusions}
	
	
	%Ce projet nous a permis de travailler sur ce qu'est un interprète. Dans %le principe c'était très interressant mais probablement trop ambitieux %et nous avons beaucoup perdu de temps a chercher se qui était attendu de %nous et a revenir sur nos pas qu'as chercher comment amélioré. Nous %n'avons donc pas pu nous attaquer au problème épineux d'écrire un %interpreteur LISP en LISP que nous pourrions interpreter grâce a notre %interpreteur en C++.
	
	Nous avons pu, à travers ce projet, appréhender le concept d'interprétation, et avoir un aperçu du fonctionnement d'un interpréteur. Le choix d'un interprète Lisp était judicieux pour ce projet, puisqu'il s'agit d'un langage relativement ancien, avec somme toute peu de contenu.

\section{Remerciements}
	Nous souhaiterions principalement remercier Luc Bougé, organisateur du projet et fournisseur de l'interprète écrit en CamL ainsi que du code de base, pour nous avoir proposé ce projet, nous avoir épargné la lourde tâche de gérer le \textit{lexer} et le \textit{parser}, ainsi que pour nous avoir suivi et aidé tout au long du projet.
	
\section{References}
http://stackoverflow.com/questions/9655202/how-to-convert-integer-to-string-in-c


http://piazza.com/ens-rennes.fr


\end{document}

